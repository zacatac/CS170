\documentclass[10pt,landscape]{article}
\usepackage{multicol}
\usepackage{calc}
\usepackage{ifthen,graphicx}
\usepackage[landscape]{geometry}
\usepackage{sidecap}
\usepackage{nonfloat}
\usepackage{lipsum}
\newenvironment{Figure}
  {\par\medskip\noindent\minipage{\linewidth}}
  {\endminipage\par\medskip}
\newcommand\myfigure[1]{%
\medskip\noindent\begin{minipage}{\columnwidth}
\centering%
#1%
%figure,caption, and label go here
\end{minipage}\medskip}
% To make this come out properly in landscape mode, do one of the following
% 1.
%  pdflatex latexsheet.tex
%
% 2.
%  latex latexsheet.tex
%  dvips -P pdf  -t landscape latexsheet.dvi
%  ps2pdf latexsheet.ps


% If you're reading this, be prepared for confusion.  Making this was
% a learning experience for me, and it shows.  Much of the placement
% was hacked in; if you make it better, let me know...


% 2008-04
% Changed page margin code to use the geometry package. Also added code for
% conditional page margins, depending on paper size. Thanks to Uwe Ziegenhagen
% for the suggestions.

% 2006-08
% Made changes based on suggestions from Gene Cooperman. <gene at ccs.neu.edu>


% To Do:
% \listoffigures \listoftables
% \setcounter{secnumdepth}{0}


% This sets page margins to .5 inch if using letter paper, and to 1cm
% if using A4 paper. (This probably isn't strictly necessary.)
% If using another size paper, use default 1cm margins.
\ifthenelse{\lengthtest { \paperwidth = 11in}}
	{ \geometry{top=.5in,left=.5in,right=.5in,bottom=.5in} }
	{\ifthenelse{ \lengthtest{ \paperwidth = 297mm}}
		{\geometry{top=1cm,left=1cm,right=1cm,bottom=1cm} }
		{\geometry{top=1cm,left=1cm,right=1cm,bottom=1cm} }
	}

% Turn off header and footer
\pagestyle{empty}
 

% Redefine section commands to use less space
\makeatletter
\renewcommand{\section}{\@startsection{section}{1}{0mm}%
                                {-1ex plus -.5ex minus -.2ex}%
                                {0.5ex plus .2ex}%x
                                {\normalfont\large\bfseries}}
\renewcommand{\subsection}{\@startsection{subsection}{2}{0mm}%
                                {-1explus -.5ex minus -.2ex}%
                                {0.5ex plus .2ex}%
                                {\normalfont\normalsize\bfseries}}
\renewcommand{\subsubsection}{\@startsection{subsubsection}{3}{0mm}%
                                {-1ex plus -.5ex minus -.2ex}%
                                {1ex plus .2ex}%
                                {\normalfont\small\bfseries}}
\makeatother

% Define BibTeX command
\def\BibTeX{{\rm B\kern-.05em{\sc i\kern-.025em b}\kern-.08em
    T\kern-.1667em\lower.7ex\hbox{E}\kern-.125emX}}

% Don't print section numbers
\setcounter{secnumdepth}{0}


\setlength{\parindent}{0pt}
\setlength{\parskip}{0pt plus 0.5ex}


% -----------------------------------------------------------------------

\begin{document}

\raggedright
\footnotesize
\begin{multicols}{3}


% multicol parameters
% These lengths are set only within the two main columns
%\setlength{\columnseprule}{0.25pt}
\setlength{\premulticols}{1pt}
\setlength{\postmulticols}{1pt}
\setlength{\multicolsep}{1pt}
\setlength{\columnsep}{2pt}

\begin{center}
     \Large{CS170 cribsheet midterm1} \\
\end{center}

\section{Order of Growth}
\subsection{Formal}
\newlength{\MyLen}
\settowidth{\MyLen}{\texttt{letterpaper}/\texttt{a4paper} \ }

UpperBound $O$ : LowerBound $\Omega$ : Constant $\Theta$
$\frac{a(n)}{b(n)}>0, a(n)\in \Omega(b(n))$
$\frac{a(n)}{b(n)}<c, a(n)\in O(b(n))$
$\frac{a(n)}{b(n)}=c, a(n)\in \Theta(b(n))$

\subsection{Tricks}
\settowidth{\MyLen}{\texttt{letterpaper}/\texttt{a4paper} \ }
$7^{\log(n)^2}=(2^{\log(7)})^{(\log(n))^2}=(2^{\log(n)})^{\log(7)\log(n)}\approx n^{\log(n)}$

$n! = 2^{n\log(n)}$ ::$36^5=6^{10}$ 

Solve the comparison by integration.
 
$(a+bi)*(c+di)\rightarrow r=ab,s=bd,t=(a+b)(c+d) = r-s+(t-r-s)i$

\subsection{add/multiply}

Karatsuba's $= \Theta(n^{\log_2{3}})$
\subsection{Prove}

Geom sum series: $g(n)=\frac{1-c^{n+1}}{1-c} = \frac{c^{n+1}-1}{c-1}$

Induction: $\gcd(F_{k+1},F_{k+1}) = \gcd(F_{k+1},F_{k+2}-F_{K+1}) = \gcd(F_{k+1},F_{k}) =1$

Numbers before prime $1/n$: in $O(n)$ time. Geom dist.$E[X]=\sum^{\infty}_{i=1}{i*P[X=i]}=\sum^{\infty}_{i=1}{i*(1-p)^{(i-1)}p}$ $p=probheads, i-1=tails throws$ 
$= p * dp/dt (\sum^{\infty}_{i=1}{-(1-p)^i})\rightarrow_{sums}=-1/p$ Integrate: $E[X]=p*(1/p^2)=1/p$ 

Binary Search: if $N$ is a square. Why only $\log n$ for power max? $N=q^k\rightarrow \log N=k\log \rightarrow k=\log N/\log q \leq \log N$ 

For any power: poweringoperation\{$\sum^{k}_{i=1}{in*n}=O(k^2n^2)$\} Repeat $\log n $ times to get $O(n^6)$

\subsection{Modular Arithmetic}
Quadratic residue busniess.
Fermat's theorem: $\forall 1\leq a < p: a^{p-1}\equiv1modp$ if p is prime.

Euler's Theorem: $m^{(p-1)(q-1)}\equiv 1 (mod pq)$
Multitudes: $2013^{2014}= 3^{2012+2}=(3^{503})^4*3^2=1*3^2=4 all mod 5$
$2012^{2013}=2^{2012+1}=(2^{503})^2*2^1=1*2=2 all mod 5$
$5^{170^{70}} mod 5$: take $170^{70}=4s+t$ form $170^{70}=(2*85)^{(2*35)}=(4*85^2)^{35}=0mod4$ 

Worst RSA: We know N,e,d: $k=(ed-1)/(p-1)(q-1)$, limit k$\in 1,2$ by $e=3,d<(p-1)(q-1)$ Solve two eq system for p and q modulating k, use $N=pq$.

Randomize recoverable RSA w/ $(M^e*k^e)^d mod N=MkmodN$ then multiply by $k^{-1}$

Primality testing: Doesn't catch Carmichaels. you did this for euler project already
\section{Divide and Conquer}
Master's Theorem: $T(n)=aT({n/b})+O(n^d),a>0,b>1,d\geq 0$

$O(n^d) \rightarrow d>\log_b{a}$ ::
$O(n^d\log n) \rightarrow d=\log_b{a}$ ::
$O(n^{\log_b{a}}) \rightarrow d<\log_b{a}$

Majority Element: If there is a majority element then it will be a majority element of $A_1$ or $A_2,, O(n\log n)$. Or you could use the pairing-discard approach $T(n)=T(n/2)+O(n)=O(n)$

For finding kth smallest element in array, $O(n) average, O(n^2) worst$
\begin{Figure}
 \includegraphics[scale=0.5]{selection}
\end{Figure}
% {%
% \setlength{\fboxsep}{0pt}%
% \setlength{\fboxrule}{0pt}%
% \fbox{\includegraphics[scale=0.25]{"unit_circle"}}%
% }%
\begin{Figure}
 \includegraphics[scale=0.2]{unit_circle_2}
\end{Figure}
%
% \myfigure{\includegraphics[width=.5\columnwidth]{unit_circle_2}%
% \figcaption{\emph{I am a figure caption!}}}
%

% \includegraphics[scale=0.25]{"unit_circle"}
Complex number practice: $\omega = e^{2\pi i/8},n=8,=\sqrt{2}/2 + i\sqrt{2}/2$

$\omega^7 = e^{2\pi i (7/8)}=\sqrt{2}/2 - i\sqrt{2}/2=\omega^-1,\omega^7 + \omega = \sqrt{2}$
$p(x)=x^2+1, p(\omega)=1+i,p(\omega^2)=0,p(\omega^3)=1-i$


Missing integer: Array A of numbers [0,N]. Split into N/2 and count the bits in least significant position. You know how may 1-bits to expect. If that number is spot on, missing=0, otherwise missing=1. For each of these splits and counts we downsize by N/2 $\rightarrow T(n)=T(n/2)+O(n)=O(n)$, all without bit complexity

Pareto points: Sort $O(n\log n)$ and then do  linear scan in reverse order $O(n)$



FFT: $A(x)=1+2x-x^2+3x^3$ $(x_1,x_2,x_3,x_4)=(\omega^0,\omega^1,\omega^2,\omega^3)=(1,i,-1,-i)::\omega=e^{2\pi i/n}$ In general find the nearest power of two as \emph{n}

Split into $A(x)=A_e(x)+xA_o(x)::A_e(x)=1-x,A_o(x)=2+3x$
$A_e(\omega^{2j})+\omega^iA_o(\omega^{2j})$

DFT Matrix entry: $(m,n)=\omega^{m*n}=e^{(2\pi i/n)*mn}$
Inverse DFT AMmtrix entry:  $(m,n)=(1/n)*\omega^{-m*n}=e^{-(2\pi i/n)*mn}$

\begin{Figure}
 \includegraphics[scale=0.35]{FFT_example}
\end{Figure}

\begin{Figure}
 \includegraphics[scale=0.7]{FFT}
\end{Figure}


\section{Graphs}
\subsection{Facts}
Undirec graph w/ n verts and n edges has cycle by induction.

Stongly connected: path between any two points
:: TREE EDGES $<=>$ CROSS EDGES depending on DFS

Dijkstra's: Put all edges on a list and mark distance $\infty$, $O(|V|)$ time.
%
%\begin{Figure}
% \includegraphics[scale=0.08]{tree_edges}
%\end{Figure}
%
%\begin{Figure}
%    \includegraphics[scale=0.7]{BFS}
%\end{Figure}
%
%\begin{Figure}
% \includegraphics[scale=0.3]{convolution}
%\end{Figure}
%\rule{0.3\linewidth}{0.25pt}
\scriptsize

\newpage

Kruskal's - greeedy - returns smallest edge not in cycle to be in MST

\section*{DP}

Find a substructure in the subproblems. Begin with the smallest
subproblem and show how the solution to that problem will give you
the "most", "least", "largest", "smallest", etc. and then give 
a recurrence for the possible steps before it. MORE SIMPLE THAN YOU THINK!



\section*{LP}

LPs must maximize only one objective function.

Dual and primal are equal then that is the optimum. Transpose everything
to get dual. Simply add dummy variables to all of the original equations
and summ the dummy variables in the objective. Simplex solves vertex
linear programs, but it does not gurantee integer solutions.

\includegraphics[scale=0.4]{max_flow.png}

\includegraphics[scale=0.5]{duality.png}

\includegraphics{qubit.png}


\section*{Reduce NP, NP-complete}

If you reduce A to B then you show how to solve A using B algorithm. Showing
NP-complete: first show B is in NP. Then show that applying B algorithm to
some other NP-complete problem will yield a solution. Make sure that the 
input and output processing functions run in {\bf polynomial time}.

A$\rightarrow$B is A in outer box of B. 

NP completes
SAT, 3SAT, Hamiltonian Path,

\newpage 

\includegraphics[scale=0.5]{zero_sum_1.png}

\includegraphics{zero_sum_2.png}

\includegraphics[scale=0.5]{zero_sum_3.png}


2013 Zack Field


\end{multicols}
\end{document}